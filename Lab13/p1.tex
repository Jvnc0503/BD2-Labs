\section{P1: Gestión de Productos y Pedidos}

\subsection{Colecciones y Datos}

Primero, se crearon las~\hyperref[fig:collections]{colecciones}:

\begin{minted}{javascript}
use lab13
db.createCollection("Productos")
db.createCollection("Clientes")
db.createCollection("Pedidos")
\end{minted}

Luego, se insertaron los datos de los archivos json desde \hyperref[fig:clientes]{MongoDB Compass}. Además,
se generaron datos adicionales con ChatGPT para evaluar mejor las consultas.

\subsection{Consultas}

Los resultados de estas consultas se pueden ver en la sección de \hyperref[consultas]{consultas} del anexo.

$\bullet$ \textbf{Obtener todos los clientes de Chile:} Esta consulta filtra los documentos por el subcampo \textit{país}
del campo \textit{dirección}.
\begin{minted}{javascript}
db.clientes.find({"direccion.pais": "Chile"})
\end{minted}

$\bullet$ \textbf{Obtener los pedidos de un cliente específico (por ID):} Similar a la consulta anterior, se hace
un filtrado por el campo \textit{id$\_$cliente} de los pedidos.
\begin{minted}{javascript}
db.pedidos.find({"id_cliente": 1})
\end{minted}

$\bullet$ \textbf{Obtener los pedidos con un total superior a cierto valor:} El filtrado se realiza en el campo
\textit{total$\_$pedido}, y seleccionamos los atributos más relevantes.
\begin{minted}{javascript}
db.pedidos.find(
    {"total_pedido": {$gt: 800}},
    {"id_cliente": 1, "id_pedido": 1, "total_pedido": 1, "_id": 0}
)
\end{minted}

$\bullet$ \textbf{Contar el número total de pedidos entre un rango de fechas:} Primero, realizamos un \textit{match} para seleccionar
los pedidos que cumplen con el criterio de fecha, y usamos \textit{count} para contar la cantidad de pedidos que cumplen la condicion.
\begin{minted}{javascript}
db.pedidos.aggregate([
    {$match:{"fecha_pedido": {$gte: "2024-09-17", $lt: "2024-09-22"}}},
    {$count: "total"}
])
\end{minted}

$\bullet$ \textbf{¿Cuáles son los $k$ productos más vendidos?} En esta consulta, usamos \textit{unwind} para desglosar los arreglos
de productos de cada pedido. Luego, los agrupamos por \textit{id$\_$producto}, y hallamos el total usando \textit{sum}. Para filtrar los
$k$ productos más vendidos, ordenamos por cantidad y limitamos el output.
\begin{minted}{javascript}
db.pedidos.aggregate([
    {$unwind: "$productos"},
    {$group: {
            _id: "$productos.id_producto",
            cantidad: {$sum: "$productos.cantidad"}
        }
    },
    {$sort: {"cantidad": -1}},
    {$limit: 5}
])
\end{minted}

Nota: En este ejemplo, obtenemos los 5 productos más vendidos. Podemos
reemplazar el 5 por cualquier otro número.

$\bullet$ \textbf{¿Quiénes son los clientes más frecuentes?} Para la consulta, agrupamos los pedidos por el id del cliente,
y hacemos la sumatoria de la cantidad de pedidos. Finalmente, ordenamos por cantidad de pedidos y limitamos el output.
\begin{minted}{javascript}
db.pedidos.aggregate([
    {$group: {
        _id: "$id_cliente",
        n_pedidos: {$sum: 1}
        }
    },
    {$sort: {"n_pedidos": -1}},
    {$limit: 5}
])
\end{minted}

Nota: En este ejemplo, obtenemos los 5 clientes más frecuentes.

\subsection{Reto Adicional}

Para crear los \hyperref[fig:creacionindices]{índices} en los campos \texttt{fecha\_pedido} y \texttt{id\_cliente}, usamos
las siguientes consultas:

\begin{minted}{javascript}
db.pedidos.createIndex({ fecha_pedido: 1 })
db.pedidos.createIndex({ id_cliente: 1 })
\end{minted}

Para utilizar el primer índice, podemos realizar una búsqueda de pedidos en un rango de fechas:

\begin{minted}{javascript}
db.pedidos.find({
    "fecha_pedido": { $gte: "2024-09-15", $lt: "2024-09-21" }
})
\end{minted}

Y para el segundo índice, buscar los pedidos realizados por un cliente en particular:

\begin{minted}{javascript}
db.pedidos.find({
    "id_cliente": 3
})
\end{minted}